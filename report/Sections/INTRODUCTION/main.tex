\section{Introduction}

The "PDM-to-Ethernet-Packet-Generator" project represents a comprehensive exploration into digital signal processing and network communication on the Digilent Nexys4-DDR FPGA Development Board. The primary objective of this project is to develop a robust system capable of capturing audio data from the MEMS microphone integrated into the FPGA board, processing the signal using the Fast Fourier Transform (FFT) algorithm, and transmitting the results via the Ethernet protocol to a host computer. This multifaceted project involved the implementation of an Ethernet frame state machine on the FPGA, enabling seamless communication between the embedded system and the external environment.

The project's significance lies in its integration of hardware and software components to achieve real-time audio signal processing and transmission. By harnessing the capabilities of the FPGA, we aimed to demonstrate the practical application of digital signal processing techniques in the context of audio data. The utilization of the Ethernet protocol facilitated efficient and reliable communication between the FPGA and the host computer, paving the way for potential applications in audio processing and analysis.

The project covers the development of the Ethernet frame state machine, the integration of the MEMS microphone for audio capture, the implementation of the FFT algorithm for signal processing, and the transmission of processed data using the Ethernet protocol. It also involves the post-processing steps undertaken on the received data, including packet capture using Wireshark and visualization using Python. 


% The collective insights from this project contribute to the broader understanding of FPGA-based signal processing systems and their applications in audio data communication.